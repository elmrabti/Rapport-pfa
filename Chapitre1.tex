\chapter{Présentation du projet}

\section{Introduction}

Dans ce chapitre nous allons mettre le sujet dans son cadre général,le cahier de charge  et l’objectif de ce projet, ainsi pour obtenir une idée sur ce que va réaliser le système en termes de métier (comportement du système). Par la suite nous aborderons la méthode adoptée pour réaliser ce
projet.


\section{Problématique}


Face aux difficultés multiples que rencontre le personnel de la santé publique dans la gestion
 et l’exploitation des dossiers médicaux des patients, face aux flux de patients et compte tenu de l’importance que représente un dossier médical, l’informatisation de ce dossier présente un
enjeu stratégique majeur.
 Elle aura beaucoup de retombées sur la qualité des soins aux seins des établissements de santé d’où l’utilité et la nécessité d’un système capable de centraliser et capitaliser l’ensemble des informations générées par le patient.
 
 
\section{Description du projet}

Pour parer à ces difficultés, un système souple et sécurisé est nécessaire.
 C’est dans ce cadre que s’inscrit notre projet de fin d’année réalisé .Il s’agit d’une
Implémentation et conception d’une application web qui permet la sauvegarde, la gestion
et la manipulation des données médicaux relatives aux traitements subis par un patient.



\section{Objectifs}


L’objectif du projet est de concevoir un application web qui consiste à mémoriser
pour chaque patient, non seulement les informations administratives (âge, sexe, adresse,
contact,…), mais également des informations médicales (le diagnostic,les ordonances,
les comptes-rendus, les traitements administrés, les analyses et les résultats ,..) et tout
autre type d’actes médicaux … puisqu'il entend améliorer la saisie, la sauvegarde
et la communication des informations de santé, tout en respectant les droits du patient bien sûr , afin de garantir une gestion souple du dossier médical de ce patient.

 L’objectif sera également d’avoir une approche originale et différente par rapport au
 sujet et d’utiliser au mieux les outils dont nous disposons pour mettre au point ce projet.
La démarche que nous avons suivie pour mettre au point cette application est simple.
 Nous avons divisé notre travail en deux différentes phases:
 \begin{itemize}
 \item Une phase où nous avons analysé le sujet, décider une approche crédible pour modéliser
l’application.
 \item une phase de développement ou nous avons commencé à mettre au point l’application en créant la base de données et le site web en question.
 \end{itemize}
 

\section{Cahier de charge}

\subsection{Introduction}

Il s’agit de la conception d’un système qui gère les données des patients dans un système de santé multi-centres, il consiste à mémoriser pour chaque patient, non seulement les informations administratives (nom, âge, sexe ...), mais également des informations concernant l’historique médical du patient, ce qui aide les médecins consultés à prendre des décisions en se basant sur des données exactes et pertinentes, en offrant des interfaces personnalisées selon le profil du demandeur de l’information, et accessible à tout moment.\\
Pour cela on propose les utilisateurs suivants : 

\begin{itemize}
\item Patient
\item Médecin
\item Radiologue
\item Admin
\end{itemize}
 





\section{Conclusion}
Dans ce chapitre nous avons pu tracer le cadre générale du projet, pour préparer la surface pour l'analyse et la conception abordés dans le chapitre suivant.



